% Entregável 4 (E4) em 12/2018: primeiro protótipo relacionado à implementação de técnicas de reconhecimento de texto. Este protótipo contemplará soluções encontradas em bibliotecas públicas, validadas no contexto de dispositivos com restrição de processamento.

\begin{abstract}

In this technical report, we present a prototype that encompasses traditional and recent methods proposed for text spotting and recognition problems, considering both non-deep and deep-learning-based approaches. We also present architectural and functional overviews of the prototype, as well as describe deployment procedures and possible usage scenarios. This report is related to the ongoing project named Multi-Lingual Text Spotting and Recognition (MLTSR), which is being developed in the context of the collaboration between Samsung R\&D Institute Brazil (SRBR) and University of Campinas (Unicamp), and it refers to the fourth deliverable (E4) planned for this project.

\end{abstract}

% Note that keywords are not normally used for peerreview papers.
\begin{IEEEkeywords}
    Text Spotting, Text Recognition, End-to-end Solutions, Scene Text Detection, Feature Engineering, Supervised Feature Learning, Convolutional Neural Networks, Recurrent Neural Network, Deep Learning, Prototype
\end{IEEEkeywords}
