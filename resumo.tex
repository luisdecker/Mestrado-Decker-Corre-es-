Múltiplas frentes de pesquisa reportaram resultados altamente eficazes para o problema de detecção de texto, que consiste no desafio de detectar em uma imagem digital a posição de variados elementos textuais, como palavras e frases. Porém, muitas destas soluções são custosas, o que restringe o uso das mesmas em várias aplicações que dependem de dispositivos com capacidade computacional restrita, como relógios inteligentes e celulares. A localização de texto é um passo importante para várias aplicações importantes que podem ser executadas em ambientes embarcados, como tradução de textos e auxílio a deficientes visuais. Neste trabalho, tratamos deste problema a partir da investigação da possibilidade do uso de redes neurais eficientes usualmente empregadas para detecção de objetos. Propusemos a junção de duas arquiteturas leves, {\em MobilenetV2} e {\em Single Shot Detector (SSD)} em nossa proposta nomeada MobText para resolver o problema da detecção de texto. Resultados experimentais nos conjuntos de dados ICDAR'11 e ICDAR'13 demonstram que nossa proposta está associada a bons resultados tanto em termos de eficácia quanto de eficiência. Em especial, o método proposto obteve resultados estado-da-arte no conjunto de dados ICDAR'11, com {\em f-measure} de $96,09\%$, mantendo um tempo de processamento médio de $464 ms$ em um ambiente de processamento restritivo. Uma outra contribuição do trabalho consistiu na proposta de uma ferramenta para automatizar o processo de avaliação de métodos de detecção e reconhecimento de textos em imagens de cena.


%\todo[inline]{
%Um abstract é composto pelos seguintes elementos (\url{https://www.youtube.com/watch?v=e968B1PwIbs}):
%\begin{enumerate}
%     \item Contextualização;
%     \item Gap da área;
%     \item Objetivo do trabalho;
%     \item Metodo proposto;
%     \item Resultados;
%     \item Conclusão.
% \end{enumerate}
% Pontos a serem melhorados neste seu abstract:
% \begin{enumerate}
%     \item Contextualização: Precisa ser expandido.
%     Sugestão: Comece falando o que é detecção de texto e sua importância; e em qual grande %área seu trabalho se insere.
%     \item Gap da área: Está OK
%     \item Objetivo do trabalho: o que você está considerando para decidir se algo é %eficiente, ou não? Tempo? Memoria? FLOPS?
%     \item Método proposto: Está OK;
%     \item Resultados: Mostre um resultado relacionado a eficiencia também;
%     \item Conclusão: Escreva um ou dois parágrafos sobre as principais conclusões do seu %trabalho.
% \end{enumerate}
%}